\hypertarget{haskell-data-types-pattern-matching-type-classes}{%
\section{Haskell Data Types, Pattern Matching, Type
Classes}\label{haskell-data-types-pattern-matching-type-classes}}

Haskell provides a way to represent and work with shapes using data
types. In this lecture, we will explore how to use Haskell data types
and compare functional programming with object-oriented programming.

\hypertarget{a-haskell-solution-using-data-types}{%
\paragraph{A Haskell Solution Using Data
Types}\label{a-haskell-solution-using-data-types}}

First, let's define a data type \texttt{Shape} that can represent
circles, squares, and right triangles:

\begin{Shaded}
\begin{Highlighting}[]
\KeywordTok{data} \DataTypeTok{Shape} 
  \OtherTok{=} \DataTypeTok{Circle}         \DataTypeTok{Double} \DataTypeTok{Double} \DataTypeTok{Double}
  \OperatorTok{|} \DataTypeTok{Square}         \DataTypeTok{Double} \DataTypeTok{Double} \DataTypeTok{Double}
  \OperatorTok{|} \DataTypeTok{RightTriangle}  \DataTypeTok{Double} \DataTypeTok{Double} \DataTypeTok{Double} \DataTypeTok{Double}
  \KeywordTok{deriving} \DataTypeTok{Show}
\end{Highlighting}
\end{Shaded}

Next, we'll define an \texttt{area} function that computes the area of a
shape:

\begin{Shaded}
\begin{Highlighting}[]
\OtherTok{area ::} \DataTypeTok{Shape} \OtherTok{{-}\textgreater{}} \DataTypeTok{Double}
\NormalTok{area shape }\OtherTok{=} 
  \KeywordTok{case}\NormalTok{ shape }\KeywordTok{of}
    \DataTypeTok{Circle}\NormalTok{ x y r          }\OtherTok{{-}\textgreater{}} \FunctionTok{pi} \OperatorTok{*}\NormalTok{ r }\OperatorTok{*}\NormalTok{ r}
    \DataTypeTok{Square}\NormalTok{ x y s          }\OtherTok{{-}\textgreater{}}\NormalTok{ s }\OperatorTok{*}\NormalTok{ s}
    \DataTypeTok{RightTriangle}\NormalTok{ x y l w }\OtherTok{{-}\textgreater{}}\NormalTok{ l }\OperatorTok{*}\NormalTok{ w}
\end{Highlighting}
\end{Shaded}

Finally, we'll define a \texttt{shift} function that shifts a shape by a
given amount in the x and y directions:

\begin{Shaded}
\begin{Highlighting}[]
\OtherTok{shift ::} \DataTypeTok{Shape} \OtherTok{{-}\textgreater{}} \DataTypeTok{Double} \OtherTok{{-}\textgreater{}} \DataTypeTok{Double} \OtherTok{{-}\textgreater{}} \DataTypeTok{Shape}
\NormalTok{shift shape delta\_x delta\_y }\OtherTok{=} 
  \KeywordTok{case}\NormalTok{ shape }\KeywordTok{of} 
    \DataTypeTok{Circle}\NormalTok{ x y r }\OtherTok{{-}\textgreater{}} \DataTypeTok{Circle}\NormalTok{ (x }\OperatorTok{+}\NormalTok{ delta\_x) (y }\OperatorTok{+}\NormalTok{ delta\_y) r}
    \DataTypeTok{Square}\NormalTok{ x y s }\OtherTok{{-}\textgreater{}} \DataTypeTok{Square}\NormalTok{ (x }\OperatorTok{+}\NormalTok{ delta\_x) (y }\OperatorTok{+}\NormalTok{ delta\_y) s}
    \DataTypeTok{RightTriangle}\NormalTok{ x y l w }\OtherTok{{-}\textgreater{}} 
    \DataTypeTok{RightTriangle}\NormalTok{ (x }\OperatorTok{+}\NormalTok{ delta\_x) (y }\OperatorTok{+}\NormalTok{ delta\_y) l w}
\end{Highlighting}
\end{Shaded}

\hypertarget{functional-programming-vs-object-oriented-programming}{%
\paragraph{Functional Programming vs Object Oriented
Programming}\label{functional-programming-vs-object-oriented-programming}}

When comparing functional programming (FP) and object-oriented
programming (OOP), it's important to understand that they have different
points of view.

In OOP, objects know how to perform different operations on themselves.
For example, a circle knows how to compute its own area and perform a
shift.

In FP, functions know how to compute over different data types. For
instance, the \texttt{shift} function knows how to compute a shifted
square, circle, or right triangle.

These two points of view are orthogonal:

\begin{itemize}
\tightlist
\item
  Object-Oriented Programming: An object knows how to perform different
  operations on itself.
\item
  Functional Programming: A function knows how to compute over different
  data types.
\end{itemize}

The key takeaway is that Haskell data types are not classes. Instead,
they provide a way to represent and work with data in a functional
programming style.

\hypertarget{simple-data-types-in-haskell}{%
\subsubsection{Simple Data Types in
Haskell}\label{simple-data-types-in-haskell}}

Haskell provides a variety of simple data types, some of which include:

\begin{longtable}[]{@{}lll@{}}
\toprule
Value & Type & Description\tabularnewline
\midrule
\endhead
1, 2, 100000000000, -42, \ldots{} & Integer & Integer
numbers\tabularnewline
3.14, 3.2831, -2.718, \ldots{} & Double & Floating-point
numbers\tabularnewline
True, False & Bool & Boolean values\tabularnewline
`a', `z', \ldots{} & Char & Character values\tabularnewline
``hello'', ``world'' & String & String values\tabularnewline
{[}1, 2, 3, 4{]} & {[}Integer{]} & List of integers\tabularnewline
(True, 0.5) & (Bool, Double) & Tuple of Bool and Double\tabularnewline
\bottomrule
\end{longtable}

All type names in Haskell are upper-case, but not all upper-case names
are types.

\hypertarget{data-type-simple-examples}{%
\paragraph{Data Type Simple Examples}\label{data-type-simple-examples}}

\begin{Shaded}
\begin{Highlighting}[]
\KeywordTok{data} \DataTypeTok{CoinFlip} \OtherTok{=} \DataTypeTok{Heads} \OperatorTok{|} \DataTypeTok{Tails}
\end{Highlighting}
\end{Shaded}

\texttt{Heads} and \texttt{Tails} are values that have type
\texttt{CoinFlip}.

\begin{Shaded}
\begin{Highlighting}[]
\KeywordTok{data} \DataTypeTok{CardSuit} \OtherTok{=} \DataTypeTok{Clubs} \OperatorTok{|} \DataTypeTok{Diamonds} \OperatorTok{|} \DataTypeTok{Hearts} \OperatorTok{|} \DataTypeTok{Spades}
\end{Highlighting}
\end{Shaded}

\texttt{Clubs}, \texttt{Diamonds}, \texttt{Hearts}, and \texttt{Spades}
are values that have type \texttt{CardSuit}.

\begin{Shaded}
\begin{Highlighting}[]
\KeywordTok{data} \DataTypeTok{ThermostatSetting} \OtherTok{=} \DataTypeTok{Off} \OperatorTok{|} \DataTypeTok{Cooling} \OperatorTok{|} \DataTypeTok{Heating}
\end{Highlighting}
\end{Shaded}

\texttt{Off}, \texttt{Cooling}, and \texttt{Heating} are values that
have type \texttt{ThermostatSetting}.

\begin{Shaded}
\begin{Highlighting}[]
\KeywordTok{data} \DataTypeTok{Bool} \OtherTok{=} \DataTypeTok{True} \OperatorTok{|} \DataTypeTok{False}
\end{Highlighting}
\end{Shaded}

\texttt{True} and \texttt{False} are values that have type
\texttt{Bool}.

\hypertarget{functions-for-data-types-simple-example}{%
\section{Functions for Data Types: Simple
Example}\label{functions-for-data-types-simple-example}}

We have a simple \texttt{ThermostatSetting} data type defined as
follows:

\begin{Shaded}
\begin{Highlighting}[]
\KeywordTok{data} \DataTypeTok{ThermostatSetting} 
\OtherTok{=} \DataTypeTok{Off} 
    \OperatorTok{|} \DataTypeTok{Cooling} 
    \OperatorTok{|} \DataTypeTok{Heating}
    \KeywordTok{deriving} \DataTypeTok{Show}
\end{Highlighting}
\end{Shaded}

Now, let's create a function \texttt{isRunning} that takes a
\texttt{ThermostatSetting} and returns a \texttt{Bool}:

\begin{Shaded}
\begin{Highlighting}[]
\OtherTok{isRunning ::} \DataTypeTok{ThermostatSetting} \OtherTok{{-}\textgreater{}} \DataTypeTok{Bool}
\end{Highlighting}
\end{Shaded}

\hypertarget{defining-functions-for-data-types}{%
\subsection{Defining Functions for Data
Types}\label{defining-functions-for-data-types}}

\hypertarget{style-1-pattern-matching-the-function-arguments}{%
\subsubsection{Style 1: Pattern Matching the Function
Arguments}\label{style-1-pattern-matching-the-function-arguments}}

\begin{Shaded}
\begin{Highlighting}[]
\OtherTok{isRunning ::} \DataTypeTok{ThermostatSetting} \OtherTok{{-}\textgreater{}} \DataTypeTok{Bool}

\NormalTok{isRunning }\DataTypeTok{Off}      \OtherTok{=} \DataTypeTok{False}
\NormalTok{isRunning }\DataTypeTok{Cooling}  \OtherTok{=} \DataTypeTok{True}
\NormalTok{isRunning }\DataTypeTok{Heating}  \OtherTok{=} \DataTypeTok{True}
\end{Highlighting}
\end{Shaded}

\hypertarget{style-2-pattern-matching-inside-a-case-expression-in-the-function-body}{%
\subsubsection{Style 2: Pattern Matching Inside a Case Expression in the
Function
Body}\label{style-2-pattern-matching-inside-a-case-expression-in-the-function-body}}

\begin{Shaded}
\begin{Highlighting}[]
\OtherTok{isRunning ::} \DataTypeTok{ThermostatSetting} \OtherTok{{-}\textgreater{}} \DataTypeTok{Bool}

\NormalTok{isRunning setting }\OtherTok{=} 
  \KeywordTok{case}\NormalTok{ setting }\KeywordTok{of}
    \DataTypeTok{Off}     \OtherTok{{-}\textgreater{}} \DataTypeTok{False}
    \DataTypeTok{Cooling} \OtherTok{{-}\textgreater{}} \DataTypeTok{True}
    \DataTypeTok{Heating} \OtherTok{{-}\textgreater{}} \DataTypeTok{True}
\end{Highlighting}
\end{Shaded}

\hypertarget{style-3-pattern-matching-the-function-arguments-with-a-wildcard-_}{%
\subsubsection{Style 3: Pattern Matching the Function Arguments with a
Wildcard
(\_)}\label{style-3-pattern-matching-the-function-arguments-with-a-wildcard-_}}

\begin{Shaded}
\begin{Highlighting}[]
\OtherTok{isRunning ::} \DataTypeTok{ThermostatSetting} \OtherTok{{-}\textgreater{}} \DataTypeTok{Bool}

\NormalTok{isRunning }\DataTypeTok{Off} \OtherTok{=} \DataTypeTok{False}
\NormalTok{isRunning \_   }\OtherTok{=} \DataTypeTok{True}
\end{Highlighting}
\end{Shaded}

\hypertarget{style-4-pattern-matching-in-a-case-expression-in-the-function-body-with-a-wildcard}{%
\subsubsection{Style 4: Pattern Matching in a Case Expression in the
Function Body with a
Wildcard}\label{style-4-pattern-matching-in-a-case-expression-in-the-function-body-with-a-wildcard}}

\begin{Shaded}
\begin{Highlighting}[]
\OtherTok{isRunning ::} \DataTypeTok{ThermostatSetting} \OtherTok{{-}\textgreater{}} \DataTypeTok{Bool}

\NormalTok{isRunning setting }\OtherTok{=} 
  \KeywordTok{case}\NormalTok{ setting }\KeywordTok{of}
    \DataTypeTok{Off} \OtherTok{{-}\textgreater{}} \DataTypeTok{False}
\NormalTok{    \_   }\OtherTok{{-}\textgreater{}} \DataTypeTok{True}
\end{Highlighting}
\end{Shaded}

\hypertarget{style-5-if-then-else-expression-in-the-function-body}{%
\subsubsection{Style 5: If-Then-Else Expression in the Function
Body}\label{style-5-if-then-else-expression-in-the-function-body}}

\begin{Shaded}
\begin{Highlighting}[]
\OtherTok{isRunning ::} \DataTypeTok{ThermostatSetting} \OtherTok{{-}\textgreater{}} \DataTypeTok{Bool}

\NormalTok{isRunning setting }\OtherTok{=} 
  \KeywordTok{if}\NormalTok{ setting }\OperatorTok{==} \DataTypeTok{Off}
  \KeywordTok{then} \DataTypeTok{False}
  \KeywordTok{else} \DataTypeTok{True}
\end{Highlighting}
\end{Shaded}

In order to use the \texttt{==} operator, we need to derive \texttt{Eq}
for the \texttt{ThermostatSetting} data type:

\begin{Shaded}
\begin{Highlighting}[]
\KeywordTok{data} \DataTypeTok{ThermostatSetting} 
\OtherTok{=} \DataTypeTok{Off} 
    \OperatorTok{|} \DataTypeTok{Cooling} 
    \OperatorTok{|} \DataTypeTok{Heating}
    \KeywordTok{deriving}\NormalTok{ (}\DataTypeTok{Show}\NormalTok{, }\DataTypeTok{Eq}\NormalTok{)}

\OtherTok{isRunning ::} \DataTypeTok{ThermostatSetting} \OtherTok{{-}\textgreater{}} \DataTypeTok{Bool}
\NormalTok{isRunning setting }\OtherTok{=} 
  \KeywordTok{if}\NormalTok{ setting }\OperatorTok{==} \DataTypeTok{Off}
  \KeywordTok{then} \DataTypeTok{False}
  \KeywordTok{else} \DataTypeTok{True}
\end{Highlighting}
\end{Shaded}

\hypertarget{style-6-a-one-liner-that-does-all-the-work-in-a-single-expression}{%
\subsubsection{Style 6: A ``One-Liner'' That Does All the Work in a
Single
Expression}\label{style-6-a-one-liner-that-does-all-the-work-in-a-single-expression}}

\begin{Shaded}
\begin{Highlighting}[]
\OtherTok{isRunning ::} \DataTypeTok{ThermostatSetting} \OtherTok{{-}\textgreater{}} \DataTypeTok{Bool}

\NormalTok{isRunning setting }\OtherTok{=}\NormalTok{ (setting }\OperatorTok{/=} \DataTypeTok{Off}\NormalTok{)}
\end{Highlighting}
\end{Shaded}

\hypertarget{style-7-a-one-liner-using-currying-with}{%
\subsubsection{Style 7: A ``One-Liner'' Using Currying with
(/=)}\label{style-7-a-one-liner-using-currying-with}}

\begin{Shaded}
\begin{Highlighting}[]
\OtherTok{isRunning ::} \DataTypeTok{ThermostatSetting} \OtherTok{{-}\textgreater{}} \DataTypeTok{Bool}

\NormalTok{isRunning }\OtherTok{=}\NormalTok{ ((}\OperatorTok{/=}\NormalTok{) }\DataTypeTok{Off}\NormalTok{)}
\end{Highlighting}
\end{Shaded}

\hypertarget{discussion}{%
\subsection{Discussion}\label{discussion}}

Which style do you personally prefer? Are there situations in which one
way is definitely better or worse than another? Consider readability,
writability, and maintainability when evaluating each style.

In general, you might not be able to tell the difference between
different implementations just by calling the \texttt{isRunning}
function, but the choice of implementation style can impact how easy it
is to understand and maintain the code. Choose a style that best suits
your preferences and the specific situation.

\hypertarget{data-types-and-associated-values}{%
\section{Data Types and Associated
Values}\label{data-types-and-associated-values}}

\hypertarget{simple-data-type-examples}{%
\subsection{Simple Data Type Examples}\label{simple-data-type-examples}}

Here are some simple data type examples:

\begin{Shaded}
\begin{Highlighting}[]
\KeywordTok{data} \DataTypeTok{CoinFlip} \OtherTok{=} \DataTypeTok{Heads} \OperatorTok{|} \DataTypeTok{Tails} \KeywordTok{deriving}\NormalTok{ (}\DataTypeTok{Show}\NormalTok{, }\DataTypeTok{Eq}\NormalTok{)}

\KeywordTok{data} \DataTypeTok{Suit} \OtherTok{=} \DataTypeTok{Clubs} \OperatorTok{|} \DataTypeTok{Diamonds} \OperatorTok{|} \DataTypeTok{Hearts} \OperatorTok{|} \DataTypeTok{Spades}
       \KeywordTok{deriving}\NormalTok{ (}\DataTypeTok{Show}\NormalTok{, }\DataTypeTok{Eq}\NormalTok{)}

\KeywordTok{data} \DataTypeTok{ThermostatSetting} 
\OtherTok{=} \DataTypeTok{Off} 
    \OperatorTok{|} \DataTypeTok{Cooling} 
    \OperatorTok{|} \DataTypeTok{Heating}
    \KeywordTok{deriving}\NormalTok{ (}\DataTypeTok{Show}\NormalTok{, }\DataTypeTok{Eq}\NormalTok{)}
\end{Highlighting}
\end{Shaded}

\hypertarget{data-types-with-associated-values}{%
\subsection{Data Types with Associated
Values}\label{data-types-with-associated-values}}

Let's consider a more complex \texttt{ThermostatSetting} data type:

\begin{Shaded}
\begin{Highlighting}[]
\KeywordTok{data} \DataTypeTok{ThermostatSetting} 
\OtherTok{=} \DataTypeTok{Off} 
    \OperatorTok{|} \DataTypeTok{CoolTo} \DataTypeTok{Int} 
    \OperatorTok{|} \DataTypeTok{HeatTo} \DataTypeTok{Int}
    \OperatorTok{|} \DataTypeTok{OutOfService} \DataTypeTok{String}
    \KeywordTok{deriving}\NormalTok{ (}\DataTypeTok{Show}\NormalTok{, }\DataTypeTok{Eq}\NormalTok{)}
\end{Highlighting}
\end{Shaded}

Some example values of the \texttt{ThermostatSetting} data type are:

\begin{verbatim}
Off                           is a value of type     ThermostatSetting
CoolTo 27                     is a value of type     ThermostatSetting
HeatTo 35                     is a value of type     ThermostatSetting
OutOfService "Maintenance"    is a value of type     ThermostatSetting
\end{verbatim}

You can create instances of these values:

\begin{Shaded}
\begin{Highlighting}[]
\NormalTok{setting\_1 }\OtherTok{=} \DataTypeTok{Off}
\NormalTok{setting\_2 }\OtherTok{=} \DataTypeTok{CoolTo} \DecValTok{20}
\NormalTok{setting\_3 }\OtherTok{=} \DataTypeTok{OutOfService} \StringTok{"Under repair"}
\end{Highlighting}
\end{Shaded}

These constructors have the following types:

\begin{verbatim}
Off             is a value of type             ThermostatSetting
CoolTo          is a (function) value of type  Int -> ThermostatSetting
HeatTo          is a (function) value of type  Int -> ThermostatSetting
OutOfService    is a (function) value of type  String -> ThermostatSetting
\end{verbatim}

\hypertarget{general-haskell-data-types}{%
\subsection{General Haskell Data
Types}\label{general-haskell-data-types}}

A Haskell datatype declaration has the following form:

\begin{Shaded}
\begin{Highlighting}[]
\KeywordTok{data} \OperatorTok{\textless{}}\DataTypeTok{TypeName}\OperatorTok{\textgreater{}} 
    \OtherTok{=} \OperatorTok{\textless{}}\DataTypeTok{ConstructorName}\OperatorTok{\textgreater{}} \OperatorTok{\textless{}}\DataTypeTok{Type}\OperatorTok{\textgreater{}} \OperatorTok{\textless{}}\DataTypeTok{Type}\OperatorTok{\textgreater{}}\NormalTok{ … }\OperatorTok{\textless{}}\DataTypeTok{Type}\OperatorTok{\textgreater{}}
    \OperatorTok{|} \OperatorTok{\textless{}}\DataTypeTok{ConstructorName}\OperatorTok{\textgreater{}} \OperatorTok{\textless{}}\DataTypeTok{Type}\OperatorTok{\textgreater{}} \OperatorTok{\textless{}}\DataTypeTok{Type}\OperatorTok{\textgreater{}}\NormalTok{ … }\OperatorTok{\textless{}}\DataTypeTok{Type}\OperatorTok{\textgreater{}}
    \OperatorTok{...}
    \OperatorTok{|} \OperatorTok{\textless{}}\DataTypeTok{ConstructorName}\OperatorTok{\textgreater{}} \OperatorTok{\textless{}}\DataTypeTok{Type}\OperatorTok{\textgreater{}} \OperatorTok{\textless{}}\DataTypeTok{Type}\OperatorTok{\textgreater{}}\NormalTok{ … }\OperatorTok{\textless{}}\DataTypeTok{Type}\OperatorTok{\textgreater{}}
\end{Highlighting}
\end{Shaded}

You can also add type class constraints using the \texttt{deriving}
keyword:

\begin{Shaded}
\begin{Highlighting}[]
\KeywordTok{data} \OperatorTok{\textless{}}\DataTypeTok{TypeName}\OperatorTok{\textgreater{}} 
    \OtherTok{=} \OperatorTok{\textless{}}\DataTypeTok{ConstructorName}\OperatorTok{\textgreater{}} \OperatorTok{\textless{}}\DataTypeTok{Type}\OperatorTok{\textgreater{}} \OperatorTok{\textless{}}\DataTypeTok{Type}\OperatorTok{\textgreater{}}\NormalTok{ … }\OperatorTok{\textless{}}\DataTypeTok{Type}\OperatorTok{\textgreater{}}
    \OperatorTok{|} \OperatorTok{\textless{}}\DataTypeTok{ConstructorName}\OperatorTok{\textgreater{}} \OperatorTok{\textless{}}\DataTypeTok{Type}\OperatorTok{\textgreater{}} \OperatorTok{\textless{}}\DataTypeTok{Type}\OperatorTok{\textgreater{}}\NormalTok{ … }\OperatorTok{\textless{}}\DataTypeTok{Type}\OperatorTok{\textgreater{}}
    \OperatorTok{...}
    \OperatorTok{|} \OperatorTok{\textless{}}\DataTypeTok{ConstructorName}\OperatorTok{\textgreater{}} \OperatorTok{\textless{}}\DataTypeTok{Type}\OperatorTok{\textgreater{}} \OperatorTok{\textless{}}\DataTypeTok{Type}\OperatorTok{\textgreater{}}\NormalTok{ … }\OperatorTok{\textless{}}\DataTypeTok{Type}\OperatorTok{\textgreater{}}
    \KeywordTok{deriving}\NormalTok{ (}\OperatorTok{\textless{}}\DataTypeTok{TypeClass1}\OperatorTok{\textgreater{}}\NormalTok{, }\OperatorTok{\textless{}}\DataTypeTok{TypeClass2}\OperatorTok{\textgreater{}}\NormalTok{, …)}
\end{Highlighting}
\end{Shaded}

\texttt{Show} and \texttt{Eq} are examples of type classes.

\hypertarget{data-types-associated-values-and-functions}{%
\subsection{Data Types, Associated Values, and
Functions}\label{data-types-associated-values-and-functions}}

\hypertarget{functions-with-data-types-and-associated-values}{%
\subsubsection{Functions with Data Types and Associated
Values}\label{functions-with-data-types-and-associated-values}}

Here is an example of a function working with the
\texttt{ThermostatSetting} data type:

\begin{Shaded}
\begin{Highlighting}[]
\OtherTok{isRunning ::} \DataTypeTok{Int} \OtherTok{{-}\textgreater{}} \DataTypeTok{ThermostatSetting} \OtherTok{{-}\textgreater{}} \DataTypeTok{Bool}

\NormalTok{isRunning temp }\DataTypeTok{Off}                \OtherTok{=} \DataTypeTok{False}
\NormalTok{isRunning temp (}\DataTypeTok{OutOfService}\NormalTok{ msg) }\OtherTok{=} \DataTypeTok{False}
\NormalTok{isRunning temp (}\DataTypeTok{CoolTo}\NormalTok{ t)         }\OtherTok{=}\NormalTok{ temp }\OperatorTok{\textgreater{}}\NormalTok{ t}
\NormalTok{isRunning temp (}\DataTypeTok{HeatTo}\NormalTok{ t)         }\OtherTok{=}\NormalTok{ temp }\OperatorTok{\textless{}}\NormalTok{ t}
\end{Highlighting}
\end{Shaded}

Alternatively, you can use a wildcard \texttt{\_} to ignore the unneeded
values:

\begin{Shaded}
\begin{Highlighting}[]
\OtherTok{isRunning ::} \DataTypeTok{Int} \OtherTok{{-}\textgreater{}} \DataTypeTok{ThermostatSetting} \OtherTok{{-}\textgreater{}} \DataTypeTok{Bool}

\NormalTok{isRunning \_ }\DataTypeTok{Off}              \OtherTok{=} \DataTypeTok{False}
\NormalTok{isRunning \_ (}\DataTypeTok{OutOfService}\NormalTok{ \_) }\OtherTok{=} \DataTypeTok{False}
\NormalTok{isRunning temp (}\DataTypeTok{CoolTo}\NormalTok{ t)    }\OtherTok{=}\NormalTok{ temp }\OperatorTok{\textgreater{}}\NormalTok{ t}
\NormalTok{isRunning temp (}\DataTypeTok{HeatTo}\NormalTok{ t)    }\OtherTok{=}\NormalTok{ temp }\OperatorTok{\textless{}}\NormalTok{ t}
\end{Highlighting}
\end{Shaded}

\hypertarget{recursive-data-types}{%
\section{Recursive Data Types}\label{recursive-data-types}}

\hypertarget{creating-custom-lists-with-recursive-data-types}{%
\subsection{Creating Custom Lists with Recursive Data
Types}\label{creating-custom-lists-with-recursive-data-types}}

A list can be an empty list or an element and another list:

\begin{Shaded}
\begin{Highlighting}[]
\KeywordTok{data} \DataTypeTok{IntList}
    \OtherTok{=} \DataTypeTok{Empty}
    \OperatorTok{|} \DataTypeTok{Cons} \DataTypeTok{Int} \DataTypeTok{IntList} 
    \KeywordTok{deriving}\NormalTok{ (}\DataTypeTok{Show}\NormalTok{)}
\end{Highlighting}
\end{Shaded}

\hypertarget{examples-of-intlist}{%
\subsubsection{Examples of IntList}\label{examples-of-intlist}}

\begin{Shaded}
\begin{Highlighting}[]
\NormalTok{list1 }\OtherTok{=} \DataTypeTok{Empty}
\NormalTok{list2 }\OtherTok{=} \DataTypeTok{Cons} \DecValTok{6} \DataTypeTok{Empty}
\NormalTok{list3 }\OtherTok{=} \DataTypeTok{Cons} \DecValTok{10}\NormalTok{ (}\DataTypeTok{Cons} \DecValTok{20}\NormalTok{ list2)}
\NormalTok{list4 }\OtherTok{=} \DataTypeTok{Cons}\NormalTok{ (}\OperatorTok{{-}}\DecValTok{4}\NormalTok{) list3}
\NormalTok{list5 }\OtherTok{=} \DataTypeTok{Cons} \DecValTok{100}\NormalTok{ (}\DataTypeTok{Cons} \DecValTok{13}\NormalTok{ list4)}
\NormalTok{list6 }\OtherTok{=} \DataTypeTok{Cons} \DecValTok{100}\NormalTok{ (}\DataTypeTok{Cons} \DecValTok{13}\NormalTok{ list5)}
\end{Highlighting}
\end{Shaded}

\hypertarget{functions-on-intlist}{%
\subsubsection{Functions on IntList}\label{functions-on-intlist}}

\hypertarget{length}{%
\paragraph{Length}\label{length}}

\begin{Shaded}
\begin{Highlighting}[]
\OtherTok{intListLength ::} \DataTypeTok{IntList} \OtherTok{{-}\textgreater{}} \DataTypeTok{Int}

\NormalTok{intListLength }\DataTypeTok{Empty} \OtherTok{=} \DecValTok{0}
\NormalTok{intListLength (}\DataTypeTok{Cons}\NormalTok{ x xs) }\OtherTok{=} \DecValTok{1} \OperatorTok{+}\NormalTok{ intListLength xs}
\end{Highlighting}
\end{Shaded}

\hypertarget{head}{%
\paragraph{Head}\label{head}}

\begin{Shaded}
\begin{Highlighting}[]
\OtherTok{intListHead ::} \DataTypeTok{IntList} \OtherTok{{-}\textgreater{}} \DataTypeTok{Int}

\NormalTok{intListHead }\DataTypeTok{Empty} \OtherTok{=} \FunctionTok{undefined}
\NormalTok{intListHead (}\DataTypeTok{Cons}\NormalTok{ x xs) }\OtherTok{=}\NormalTok{ x}
\end{Highlighting}
\end{Shaded}

\hypertarget{tail}{%
\paragraph{Tail}\label{tail}}

\begin{Shaded}
\begin{Highlighting}[]
\OtherTok{intListTail ::} \DataTypeTok{IntList} \OtherTok{{-}\textgreater{}} \DataTypeTok{IntList}

\NormalTok{intListTail }\DataTypeTok{Empty} \OtherTok{=} \FunctionTok{undefined}
\NormalTok{intListTail (}\DataTypeTok{Cons}\NormalTok{ x xs) }\OtherTok{=}\NormalTok{ xs}
\end{Highlighting}
\end{Shaded}

\hypertarget{map}{%
\paragraph{Map}\label{map}}

\begin{Shaded}
\begin{Highlighting}[]
\OtherTok{intListMap ::}\NormalTok{ (}\DataTypeTok{Int} \OtherTok{{-}\textgreater{}} \DataTypeTok{Int}\NormalTok{) }\OtherTok{{-}\textgreater{}} \DataTypeTok{IntList} \OtherTok{{-}\textgreater{}} \DataTypeTok{IntList}

\NormalTok{intListMap f }\DataTypeTok{Empty} \OtherTok{=} \DataTypeTok{Empty}
\NormalTok{intListMap f (}\DataTypeTok{Cons}\NormalTok{ x xs) }\OtherTok{=} \DataTypeTok{Cons}\NormalTok{ (f x) (intListMap f xs)}
\end{Highlighting}
\end{Shaded}

\hypertarget{sum}{%
\paragraph{Sum}\label{sum}}

\begin{Shaded}
\begin{Highlighting}[]
\OtherTok{intListSum ::} \DataTypeTok{IntList} \OtherTok{{-}\textgreater{}} \DataTypeTok{Int}

\NormalTok{intListSum }\DataTypeTok{Empty} \OtherTok{=} \DecValTok{0}
\NormalTok{intListSum (}\DataTypeTok{Cons}\NormalTok{ x xs) }\OtherTok{=}\NormalTok{ x }\OperatorTok{+}\NormalTok{ intListSum xs}
\end{Highlighting}
\end{Shaded}

\hypertarget{string-binary-trees-with-recursive-data-types}{%
\subsection{String Binary Trees with Recursive Data
Types}\label{string-binary-trees-with-recursive-data-types}}

A \texttt{StringBinaryTree} is either a leaf with a \texttt{String} in
it or a node with a \texttt{String} and two \texttt{StringBinaryTree}s:

\begin{Shaded}
\begin{Highlighting}[]
\KeywordTok{data} \DataTypeTok{StringBinaryTree}
  \OtherTok{=} \DataTypeTok{Leaf} \DataTypeTok{String}
  \OperatorTok{|} \DataTypeTok{Node} \DataTypeTok{String} \DataTypeTok{StringBinaryTree} \DataTypeTok{StringBinaryTree}
  \KeywordTok{deriving}\NormalTok{ (}\DataTypeTok{Show}\NormalTok{)}
\end{Highlighting}
\end{Shaded}

\hypertarget{examples-of-stringbinarytree}{%
\subsubsection{Examples of
StringBinaryTree}\label{examples-of-stringbinarytree}}

\begin{Shaded}
\begin{Highlighting}[]
\NormalTok{tree1 }\OtherTok{=} \DataTypeTok{Leaf} \StringTok{"a"}
\NormalTok{tree2 }\OtherTok{=} \DataTypeTok{Leaf} \StringTok{"b"}
\NormalTok{tree3 }\OtherTok{=} \DataTypeTok{Leaf} \StringTok{"c"}
\NormalTok{tree4 }\OtherTok{=} \DataTypeTok{Node} \StringTok{"f"}\NormalTok{ (}\DataTypeTok{Leaf} \StringTok{"d"}\NormalTok{) (}\DataTypeTok{Leaf} \StringTok{"e"}\NormalTok{)}
\NormalTok{tree5 }\OtherTok{=} \DataTypeTok{Node} \StringTok{"g"}\NormalTok{ tree1 tree2}
\NormalTok{tree6 }\OtherTok{=} \DataTypeTok{Node} \StringTok{"h"}\NormalTok{ tree5 tree4}
\NormalTok{tree7 }\OtherTok{=} \DataTypeTok{Node} \StringTok{"i"}\NormalTok{ tree3 tree6}
\end{Highlighting}
\end{Shaded}

\hypertarget{functions-on-stringbinarytree}{%
\subsubsection{Functions on
StringBinaryTree}\label{functions-on-stringbinarytree}}

\hypertarget{size}{%
\paragraph{Size}\label{size}}

\begin{Shaded}
\begin{Highlighting}[]
\OtherTok{treeSize ::} \DataTypeTok{StringBinaryTree} \OtherTok{{-}\textgreater{}} \DataTypeTok{Int}

\NormalTok{treeSize (}\DataTypeTok{Leaf}\NormalTok{ \_) }\OtherTok{=} \DecValTok{1}
\NormalTok{treeSize (}\DataTypeTok{Node}\NormalTok{ \_ left right) }\OtherTok{=} \DecValTok{1} \OperatorTok{+}\NormalTok{ (treeSize left) }\OperatorTok{+}\NormalTok{ (treeSize right)}
\end{Highlighting}
\end{Shaded}

\hypertarget{height}{%
\paragraph{Height}\label{height}}

\begin{Shaded}
\begin{Highlighting}[]
\OtherTok{treeHeight ::} \DataTypeTok{StringBinaryTree} \OtherTok{{-}\textgreater{}} \DataTypeTok{Int}

\NormalTok{treeHeight (}\DataTypeTok{Leaf}\NormalTok{ \_) }\OtherTok{=} \DecValTok{0}
\NormalTok{treeHeight (}\DataTypeTok{Node}\NormalTok{ \_ left right) }\OtherTok{=} \DecValTok{1} \OperatorTok{+} \FunctionTok{max}\NormalTok{ (treeHeight left) (treeHeight right)}
\end{Highlighting}
\end{Shaded}

\hypertarget{map-1}{%
\paragraph{Map}\label{map-1}}

\begin{Shaded}
\begin{Highlighting}[]
\OtherTok{treeMap ::}\NormalTok{ (}\DataTypeTok{String} \OtherTok{{-}\textgreater{}} \DataTypeTok{String}\NormalTok{) }\OtherTok{{-}\textgreater{}} \DataTypeTok{StringBinaryTree} \OtherTok{{-}\textgreater{}} \DataTypeTok{StringBinaryTree}

\NormalTok{treeMap f (}\DataTypeTok{Leaf}\NormalTok{ s) }\OtherTok{=} \DataTypeTok{Leaf}\NormalTok{ (f s)}
\NormalTok{treeMap f (}\DataTypeTok{Node}\NormalTok{ s left right) }\OtherTok{=} \DataTypeTok{Node}\NormalTok{ (f s) (treeMap f left) (treeMap f right)}
\end{Highlighting}
\end{Shaded}

\hypertarget{example-excited-tree}{%
\subsubsection{Example: Excited Tree}\label{example-excited-tree}}

\begin{Shaded}
\begin{Highlighting}[]
\NormalTok{excitedTree }\OtherTok{=}\NormalTok{ treeMap (}\OperatorTok{++} \StringTok{"!"}\NormalTok{) tree7}
\end{Highlighting}
\end{Shaded}

\hypertarget{type-classes}{%
\section{Type Classes}\label{type-classes}}

\hypertarget{values-types-and-type-classes}{%
\subsection{Values, Types, and Type
Classes}\label{values-types-and-type-classes}}

We've already learned a lot about values and types. Values are instances
of Types, and this relationship is similar to set membership, where
values are members of types.

Example values:

\begin{itemize}
\tightlist
\item
  `C'
\item
  True
\item
  {[}True, False{]}
\item
  ``Hello''
\end{itemize}

The Types of these values, respectively, are:

\begin{itemize}
\tightlist
\item
  Char
\item
  Bool
\item
  {[}Bool{]}
\item
  {[}Char{]}
\end{itemize}

\hypertarget{haskells-families-of-types-type-classes}{%
\subsection{Haskell's Families of Types --- Type
Classes}\label{haskells-families-of-types-type-classes}}

Types belong to type classes:

\begin{itemize}
\tightlist
\item
  \texttt{Eq} defines the \texttt{==} and \texttt{/=} operators
\item
  \texttt{Ord} defines comparison operators, such as
  \texttt{\textless{}} (Note that any type in \texttt{Ord} must also be
  in \texttt{Eq})
\item
  \texttt{Read} defines the \texttt{read} operator which turns a
  \texttt{String} into a value
\item
  \texttt{Show} defines the \texttt{show} operator which turns a value
  into a \texttt{String}
\item
  \texttt{Num} defines \texttt{+}, \texttt{-}, \texttt{/}, etc.
\end{itemize}

There are many more type classes, and we'll introduce them as needed.

\hypertarget{simple-data-type-examples-1}{%
\subsubsection{Simple Data Type
Examples}\label{simple-data-type-examples-1}}

\begin{Shaded}
\begin{Highlighting}[]
\KeywordTok{data} \DataTypeTok{CardSuit} \OtherTok{=} \DataTypeTok{Clubs} \OperatorTok{|} \DataTypeTok{Diamonds} \OperatorTok{|} \DataTypeTok{Hearts} \OperatorTok{|} \DataTypeTok{Spades}
\end{Highlighting}
\end{Shaded}

Here, \texttt{Clubs}, \texttt{Diamonds}, \texttt{Hearts}, and
\texttt{Spades} are values that have the \texttt{CardSuit} type.

\hypertarget{data-type-examples-with-type-classes}{%
\subsubsection{Data Type Examples with Type
Classes}\label{data-type-examples-with-type-classes}}

\begin{Shaded}
\begin{Highlighting}[]
\KeywordTok{data} \DataTypeTok{CardSuit} \OtherTok{=} \DataTypeTok{Clubs} \OperatorTok{|} \DataTypeTok{Diamonds} \OperatorTok{|} \DataTypeTok{Hearts} \OperatorTok{|} \DataTypeTok{Spades}
    \KeywordTok{deriving}\NormalTok{ (}\DataTypeTok{Show}\NormalTok{, }\DataTypeTok{Eq}\NormalTok{, }\DataTypeTok{Ord}\NormalTok{, }\DataTypeTok{Read}\NormalTok{)}
\end{Highlighting}
\end{Shaded}

\texttt{Clubs}, \texttt{Diamonds}, \texttt{Hearts}, and \texttt{Spades}
are values that have the \texttt{CardSuit} type.

\hypertarget{type-classes-are-not-oop-classes}{%
\subsection{Type Classes are NOT OOP
Classes}\label{type-classes-are-not-oop-classes}}

Type classes in Haskell are not the same as the concept of classes from
object-oriented programming. They both use the word ``class,'' but they
mean totally different things.

\hypertarget{bonus-combining-two-or-more-data-types}{%
\subsection{Bonus: Combining Two (or more) Data
Types}\label{bonus-combining-two-or-more-data-types}}

\hypertarget{combining-data-types}{%
\subsubsection{Combining Data Types}\label{combining-data-types}}

\begin{Shaded}
\begin{Highlighting}[]
\KeywordTok{data} \DataTypeTok{CardSuit} \OtherTok{=} \DataTypeTok{Clubs} \OperatorTok{|} \DataTypeTok{Diamonds} \OperatorTok{|} \DataTypeTok{Hearts} \OperatorTok{|} \DataTypeTok{Spades}
  \KeywordTok{deriving}\NormalTok{ (}\DataTypeTok{Show}\NormalTok{, }\DataTypeTok{Eq}\NormalTok{, }\DataTypeTok{Ord}\NormalTok{)}

\KeywordTok{data} \DataTypeTok{FaceValues}
  \OtherTok{=} \DataTypeTok{Two}  \OperatorTok{|} \DataTypeTok{Three}  \OperatorTok{|} \DataTypeTok{Four}  \OperatorTok{|} \DataTypeTok{Five}  \OperatorTok{|} \DataTypeTok{Six}  \OperatorTok{|} \DataTypeTok{Seven}  \OperatorTok{|} \DataTypeTok{Eight}  
  \OperatorTok{|} \DataTypeTok{Nine}  \OperatorTok{|} \DataTypeTok{Ten}  \OperatorTok{|} \DataTypeTok{Jack} \OperatorTok{|} \DataTypeTok{Queen}  \OperatorTok{|} \DataTypeTok{King} \OperatorTok{|} \DataTypeTok{Ace}
  \KeywordTok{deriving}\NormalTok{ (}\DataTypeTok{Show}\NormalTok{, }\DataTypeTok{Eq}\NormalTok{, }\DataTypeTok{Ord}\NormalTok{)}

\KeywordTok{type} \DataTypeTok{Card} \OtherTok{=}\NormalTok{ (}\DataTypeTok{FaceValues}\NormalTok{, }\DataTypeTok{CardSuit}\NormalTok{)}

\KeywordTok{data} \DataTypeTok{CardList}
  \OtherTok{=} \DataTypeTok{Empty}
  \OperatorTok{|} \DataTypeTok{Hand} \DataTypeTok{Card} \DataTypeTok{CardList}
  \KeywordTok{deriving}\NormalTok{ (}\DataTypeTok{Show}\NormalTok{, }\DataTypeTok{Eq}\NormalTok{, }\DataTypeTok{Ord}\NormalTok{)}
\end{Highlighting}
\end{Shaded}

Example of a \texttt{CardList}:

\begin{Shaded}
\begin{Highlighting}[]
\NormalTok{cardList1 }\OtherTok{=} \DataTypeTok{Hand}\NormalTok{ (}\DataTypeTok{Two}\NormalTok{, }\DataTypeTok{Hearts}\NormalTok{) (}\DataTypeTok{Hand}\NormalTok{ (}\DataTypeTok{Ace}\NormalTok{, }\DataTypeTok{Diamonds}\NormalTok{) (}\DataTypeTok{Hand}\NormalTok{ (}\DataTypeTok{Ten}\NormalTok{, }\DataTypeTok{Spades}\NormalTok{) }\DataTypeTok{Empty}\NormalTok{))}
\end{Highlighting}
\end{Shaded}
